\input{Preamble.tex}

\newcommand{\ginebig}{{\textsf{G\raisebox{-0.3ex}{i}\raisebox{0.3ex}{n}\raisebox{-0.3ex}{e}big}}}

\lhead{\scriptsize Barnes}
\chead{\scriptsize}
\rhead{\scriptsize \thepage}

\lfoot{\scriptsize \ginebig{} User's Guide and Manual}
\cfoot{\scriptsize}
\rfoot{\scriptsize Version: \today}


%******************************************************************************
\begin{document}
%******************************************************************************

%==============================================================================
% Title block information
%==============================================================================
\title{\ginebig{}: A Single-layer AEM Library Using Python}

\author{
Dr. Randal J. Barnes\\
Department of Civil Engineering\\
University of Minnesota
}

\date{Draft: \today}

\maketitle

\thispagestyle{plain}


%==============================================================================
\section{Introduction}
%==============================================================================




\appendix
\newpage
%==============================================================================
\section{Code Naming Conventions}
%==============================================================================
The following is extracted from ``PEP 8 -- Style Guide for Python Code''.

\begin{description}
    \item [function names] Function names should be lowercase, with words separated by underscores as necessary to improve readability.

    \item [instance variables] Use the function naming rules: lowercase with words separated by underscores as necessary to improve readability.

    \item [method names] Use the function naming rules: lowercase with words separated by underscores as necessary to improve readability.

    \item [class names] Class names should normally use the {\tt CapWords} (aka {\tt UpperCamelCase}) convention.

    \item [package names] Python packages should also have short, all-lowercase names, although the use of underscores is discouraged.

    \item [module names] Modules should have short, all-lowercase names. Underscores can be used in the module name if it improves readability.

    \item [exception names] Because exceptions should be classes, the class naming convention applies here. However, you should use the suffix "Error" on your exception names (if the exception actually is an error).

    \item [constants] Constants are usually defined on a module level and written in all capital letters with underscores separating words. Examples include {\tt MAX\_OVERFLOW} and {\tt TOTAL}.

    \item [function arguments] Always use {\tt self} for the first argument to instance methods.

    \item [method arguments] Always use {\tt cls} for the first argument to class methods.


\end{description}





%******************************************************************************
\end{document}
%******************************************************************************
